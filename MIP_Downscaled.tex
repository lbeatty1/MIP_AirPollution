\documentclass[a4paper]{article}
%% Sets page size and margins
\usepackage[a4paper,top=3cm,bottom=3cm,left=3cm,right=3cm,marginparwidth=1.75cm]{geometry}
%% Useful packages
\usepackage{amsmath,amsthm,amssymb,amsfonts}
\usepackage{graphicx}
\usepackage[colorinlistoftodos]{todonotes}
\usepackage{bbm}
\usepackage{setspace}
\usepackage{footmisc}
\usepackage{pdflscape}
\usepackage{natbib}
\usepackage{booktabs}
\usepackage{caption}
\usepackage{subcaption}
\usepackage{changepage}
\usepackage{rotating}
\usepackage{bm}

\usepackage{graphicx}
\usepackage[colorlinks=true, allcolors=blue]{hyperref}
\usepackage{url}

\renewcommand\footnotelayout{\fontsize{10}{12}\selectfont}



\interfootnotelinepenalty=10000

\newcommand{\R}{\mathbb{R}}
\newcommand{\N}{\mathbb{N}}
\newcommand{\Z}{\mathbb{Z}}
\providecommand{\C}{\mathbb{C}}

\theoremstyle{definition}
\newtheorem{defin}{Definição}

\theoremstyle{plain}
\newtheorem{theorem}[defin]{Teorema}
\newtheorem{corollary}[defin]{Corolário}

\linespread{2}
\title{The Distributional Impacts of the Electricity Transition on Pollution Exposure}

\author{Lauren Beatty\thanks{Environmental Defense Fund  \hspace{.5cm} E-mail: lbeatty1@edf.org \hspace{.5cm}Website: \href{https://lbeatty1.github.io}{https://lbeatty1.github.io}}, 
Matthias Fripp, 
Greg Schivley, 
Rangrang Zhang, 
Cameron Wade, \\ 
Oleg Lugovoy, 
Michael Roberts, 
Patricia Hildalgo}


\date{\today}

\begin{document}
\maketitle
\begin{center}
    PRELIMINARY DRAFT - PLEASE DO NOT CITE OR DISTRIBUTE
\end{center}

\begin{abstract}
    In this paper, we provide an overview of how different scenarios for the electricity transition will affect the distribution of air pollution exposure through 2050.  Currently, there are large disparities in pollution exposure across racial and socio-economic groups resulting from historical choices about where generation facilities were cited as well as differences across groups in income and wealth. We use results from the Model Intercomparison Project -- a project aimed at harmonizing the inputs and comparing the outputs of multiple open-source least-cost electricity planning models.  We take those results, predict location-specific emissions of various pollutants over time, then use InMap to calculate average group-level exposures to PM2.5.  We find broad agreement across models.  Moreover, we find that deep decarbonization should lead to major improvements in air quality, and reductions in pollution exposure, along with a reduction in pollution exposure disparities.
\end{abstract}

\newpage
\section{Introduction}
Historical inequities have led to large contemporary differences in air pollution exposure across groups.  This paper looks forward and analyzes how different plans for the U.S. electricity transition will affect group-level pollution exposure. 


There is a rich literature on the effects of electricity generation on pollution, the effects of pollution on human health and well-being, and papers linking electricity generation to health effects.  Air pollution is linked with numerous health impacts such as chronic obstructive pulmonary disease (COPD), acture lower respiratory illness (ALRI), cerebrovascular disease (CEV), ischaemic heart disease (IHD), and lung cancer, and in the United States, power generation is responsible for $31\%$ of premature mortality linked to outdoor air pollution \citep{Lelieveld2015TheScale}.

There is also a rich literature on how policies affect air pollution from power generation.  \citet{Shawhan2024PoliciesAmericans} use the  Engineering, Economic, and Environmental Electricity Simulation Tool (E4ST) to model capacity expansion and system operation combined with InMap to model how policies aimed at reducing air pollution in environmentally overburdened, disadvantaged communities (EO DACs) compare with hypothetical $\$15$ per ton carbon tax.  They find that both CO$_2$ pricing and the best-performing within-DAC policies reduce mortality by approximately 1.4 to 1.5 times as much for EO DAC residences as non-EO DAC residents -- a major win for reducing disparities.  However, they find that the targeted policies are only slightly more effective than the carbon tax at reducing PM2.5, and that the plans are far more costly than a carbon tax.  Similarly, \citet{Goforth2022AirStrategies} use the Regional Energy Deployment System (ReEDS) along with InMAP to project how different U.S. national decarbonization strategies will affect air pollution disparities.  

This paper extends the literature by projecting air pollution impacts for a wider variety of policy scenarios, across multiple open-source least-cost electricity models.  Our results are in broad agreement with the studies of \citet{Goforth2022AirStrategies} and \citet{Shawhan2024PoliciesAmericans}.  We find that decarbonization strategies should lead to a large decrease in pollution exposure, as well as  a decline in disparities across groups in exposure.  Moreover, we find strong agreement between models.



\section{Data}

\subsection{Emissions Data}
Emissions data comes from two sources: EPA's Clean Air Markets Program Data (CAMD) and EPA's National Emissions Inventory (NEI).  The CAMD was created to track compliance with clean air programs, and thus, tracks emissions of CO$_2$, NO$_x$, and SO$_2$.  The NEI was created to track criteria pollutants, criteria precursors, and hazardous air pollutants from \textit{all} sources.  I get emissions of NH$_3$, VOC, and PM2.5 from the NEI. 

\subsection{Data for PowerGenome}
Data for the MIP was generated using PowerGenome, then modified on a model-by-model basis.  PowerGenome was designed to produce input files for the GenX capacity expansion model.  PowerGenome relies on a wide variety of data such as the National Renewable Energy Lab's (NREL) Annual Technology Baseline (ATB), the U.S. Energy Information Administration's Annual Energy Outlook, the EPA's Integrated Planning Model, and more.  For more information on PowerGenome, see the release page \citep{PowerGenome/PowerGenome:V0.6.3}.  Further details on the construction of model inputs for TEMOA, SWITCH, and USENSYS can be found in the main MIP paper at INSERT PAPER HERE.

\section{Methods}

\subsection{Calculating source-specific emissions rates}
\begin{enumerate}
    \item Join EIA generation data with EPA emissions data to calculate emissions rates in lbs/MWh.
    \item Calculate emissions from existing plants by taking model-outputted generation and multiplying that by the plant's share of cluster capacity to get a predicted MWh.  Then multiply that by the plant's emissions rates for all five pollutants.
    \item New fossil generation outputted by these models is only natural gas.  For new plants, I start by allocating capacity to both retired and existing sites, with a preference for siting at retired sites.
\end{enumerate}

https://ehp.niehs.nih.gov/doi/10.1289/EHP9001
https://www.rff.org/publications/working-papers/policies-for-reducing-the-impacts-of-power-sector-air-pollution-on-disadvantaged-americans/

\subsection{Least-cost modeling}
We use four different open-source least-cost electricity planning models to solve many different scenarios. For more information on the models and scenarios, see OTHER PAPER.

\subsection{Emissions to Exposure: InMap}
Once we've obtained predictions for plant-level emissions, we feed these emissions into the InMap Source-Receptor Matrix (ISRM).  The ISRM was developed to further simply the InMAP model.  The InMAP model was designed to compute annual-average primary and secondary PM2.5.  InMAP is not a full Eulerian chemical tranport model. Rather, InMAP uses information from chemical transport models along with some simplifying assumptions and estimated reduced-form parameters \citep{Tessum2017InMAP:Interventions}.  The InMAP ISRM was developed by running InMAP hundreds of thousands of times to calculate the average linear relationship between emissions in each source location with annual-average PM2.5 concentrations at receptor locations \citep{Goodkind2019Fine-scaleEmissions}.  The InMAP ISRM uses a variable grid with a finer scale in more population-dense areas, and a coarser definition in less dense areas, greatly reducing the size of the matrix while still maintaining it's ability to make fine-scale spatial prediction.  We plot an example of the ISRM output below in figure \ref{ISRMexample}.
\begin{figure}
    \centering
    \begin{subfigure}[b]{0.45\textwidth}
        \includegraphics[width=\textwidth]{Figures/Output/full-current-policies/GenX/ISRM_2027_TotalPM25_concentrationmap.jpg}
        \caption{2027}
        \label{Subfig1}
    \end{subfigure}
    \hfill
    \begin{subfigure}[b]{0.45\textwidth}
        \includegraphics[width=\textwidth]{Figures/Output/full-current-policies/GenX/ISRM_2050_TotalPM25_concentrationmap.jpg}
        \caption{2050}
        \label{Subfig2}
    \end{subfigure}\\
    \caption{ISRM Output for the Current Policies Scenario in 2027 and 2050.}
\end{figure}
We translate concentrations into mortality using total PM2.5 predicted by INMAP in a Cox proportional hazards equation:

\begin{equation}
    Deaths = [\exp(\log(r)/10*PM2.5)-1]\times Pop \times \frac{Pop_y}{Pop_{2010}} \times MortalityRate
\end{equation}
where 
\begin{itemize}
    \item $\frac{Pop_y}{Pop_{2010}}=1.0735$ is the ratio between year-$y$ population and year-2010 population (what's natively in the model).  For each model period we calculate this ratio from the 2023 U.S. Census Bureau's National Population Projection Tables.  These tables have projections by race and by year, so our estimates will reflect, for example, the fact that the Hispanic and Latino population is expected to grow by 42$\%$ by 2050, while the Non-Hispanic White population is expected to decrease \citep{Bureau2023Series}.
    \item $MortalityRate$ is the baseline mortality rate.
    \item $r$ is a parameter describing how mortality rates respond to a 10$\mu g/m^3$ increase in PM2.5 concentrations.  For this paper we use 1.06, reported by \citet{Krewski2009ExtendedInstitute}.
\end{itemize}

We project deaths forward but keep the calculation of deaths constant.  That means that our projections will not account for idiosyncratic differences across space or groups in migration patterns, or changes to the mortality rate.  Our primary concern is differences in mortality \textit{rates} rather than aggregate mortality.  Moreover, we primarily want our estimates to reflect changes in electricity generation, rather than population projections with wide uncertainty bands.
\section{Results: Comparison Across Models}

\section{Results: Comparison Across Scenarios}

Given the precise agreement between the models, we now turn our attention to comparing pollution exposure across policy scenarios.  In figure \ref{Compare Scenarios}, we plot population-weighted mean death rates by racial group by scenario for each model period.  The first thing that becomes immediately apparent is that the decarbonization scenarios reduce pollution exposure much more quickly and dramatically than the current policy scenarios.  It's also clear that there are wide disparities across racial groups, these disparities decrease in aggregate magnitudes but remain more constant in relative magnitudes over time.
\begin{figure}
    \centering
    \begin{subfigure}[b]{0.45\textwidth}
        \includegraphics[width=\textwidth]{Figures/Output/ISRM_deathrate_by_scenario_2027.jpg}
        \caption{2027}
        \label{Subfig1}
    \end{subfigure}
    \hfill
    \begin{subfigure}[b]{0.45\textwidth}
        \includegraphics[width=\textwidth]{Figures/Output/ISRM_deathrate_by_scenario_2030.jpg}
        \caption{2030}
        \label{Subfig2}
    \end{subfigure}\\

        \begin{subfigure}[b]{0.45\textwidth}
        \includegraphics[width=\textwidth]{Figures/Output/ISRM_deathrate_by_scenario_2030.jpg}
        \caption{2035}
        \label{Subfig2}
    \end{subfigure}
    \hfill
        \begin{subfigure}[b]{0.45\textwidth}
        \includegraphics[width=\textwidth]{Figures/Output/ISRM_deathrate_by_scenario_2035.jpg}
        \caption{2040}
        \label{Subfig2}
    \end{subfigure}\\

    \begin{subfigure}[b]{0.45\textwidth}
        \includegraphics[width=\textwidth]{Figures/Output/ISRM_deathrate_by_scenario_2045.jpg}
        \caption{2045}
        \label{Subfig2}
    \end{subfigure}
    \hfill
    \begin{subfigure}[b]{0.45\textwidth}
        \includegraphics[width=\textwidth]{Figures/Output/ISRM_deathrate_by_scenario_2050.jpg}
        \caption{2050}
        \label{Subfig2}
    \end{subfigure}
    \caption{Comparison results by scenario over time}
    \label{Compare Scenarios}
\end{figure}

FIGURE ON CO2 Vs Exposure

FIGURE ON TOTAL COST VS EXPOSURE
\subsection{Foresight}

\subsection{Constraining Transmission Expansion}

\section{Conclusion}

\begin{singlespace}
\newpage
\bibliographystyle{jpe}
%\bibliographystyle{econometrica}
\bibliography{references.bib}
\end{singlespace}

\end{document}