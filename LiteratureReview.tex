\documentclass[a4paper]{article}
%% Sets page size and margins
\usepackage[a4paper,top=3cm,bottom=3cm,left=3cm,right=3cm,marginparwidth=1.75cm]{geometry}
%% Useful packages
\usepackage{amsmath,amsthm,amssymb,amsfonts}
\usepackage{graphicx}
\usepackage[colorinlistoftodos]{todonotes}
\usepackage{bbm}
\usepackage{setspace}
\usepackage{footmisc}
\usepackage{pdflscape}
\usepackage{natbib}
\usepackage{booktabs}
\usepackage{caption}
\usepackage{subcaption}
\usepackage{changepage}
\usepackage{rotating}
\usepackage{bm}

\usepackage{graphicx}
\usepackage[colorlinks=true, allcolors=blue]{hyperref}
\usepackage{url}

\renewcommand\footnotelayout{\fontsize{10}{12}\selectfont}



\interfootnotelinepenalty=10000

\newcommand{\R}{\mathbb{R}}
\newcommand{\N}{\mathbb{N}}
\newcommand{\Z}{\mathbb{Z}}
\providecommand{\C}{\mathbb{C}}

\theoremstyle{definition}
\newtheorem{defin}{Definição}

\theoremstyle{plain}
\newtheorem{theorem}[defin]{Teorema}
\newtheorem{corollary}[defin]{Corolário}

\linespread{2}
\title{Air Quality and Electricity Generation: Literature Review}

\author{Lauren Beatty \thanks{Environmental Defense Fund\\  E-mail: lbeatty1@edf.org \hspace{.5cm}Website: \href{https://lbeatty1.github.io}{https://lbeatty1.github.io}}}

\date{\today}

\begin{document}
\maketitle
\begin{center}
    PRELIMINARY DRAFT - PLEASE DO NOT CITE OR DISTRIBUTE
\end{center}


\newpage
\section{Introduction}

There is a rich literature on the effects of electricity generation on pollution and the effects of pollution on human health and well-being.  


In addition, there has been an increasing focus on how researchers and public policy makers can consider the holistic, distributional effects of policies.  

Lelieveld et al.  \citep{Lelieveld2015TheScale}
- air pollution is linked with numerous health impacts such as chronic obstructive pulmonary disease (COPD), acture lower respiratory illness (ALRI), cerebrovascular disease (CEV), ischaemic heart disease (IHD), and lung cancer.
- \textit{We have calculated premature mortality linked to CEV, COPD, IHD and LC for adults 30 years old, and ALRI for infants ,5 years old (Table 1 and Extended Data Tables 1 and 2). Our estimate of the globalPM2.5 related mortality in 2010 is 3.15 million people with a 95confidence interval (CI95) of 1.52–4.60 million. The main causes are CEV (1.31 million) and IHD (1.08 million), and secondary causes are COPD (374 thousand), ALRI (230 thousand) and LC (161 thousand)}
- In the U.S., power generation is responsible for 31$\%$ of premature mortality linked to outdoor air pollution
- Calc deltaMort = y0 [(RR01)/RR]Pop
- RR=exp(b(X-X0)) where b is concentration response coefficient

\subsection{InMap}
\citep{Tessum2017InMAP:Interventions}
- Annual average changes in primary and secondary PM2.5
- Eulerian Chemical Transportation Models are computationally expensive and time consuming
- Uses information from CTM along with simplifying assumptions
- Variable grid to focus on human exposures
- Method

\subsection{Muller Work - Brief Outline of AP3}
\citep{}

\section{Description of Air Pollution MIP Modelling}

\subsection{Emissions Data}
Emissions data comes from two sources: EPA's Clean Air Markets Program Data (CAMD) and EPA's National Emissions Inventory (NEI).  The CAMD was created to track compliance with clean air programs, and thus, tracks emissions of CO$_2$, NO$_x$, and SO$_2$.  The NEI was created to track criteria pollutants, criteria precursors, and hazardous air pollutants from \textit{all} sources.  I get emissions of NH$_3$, VOC, and PM2.5 from the NEI. 

\subsection{High-level outline}
\begin{enumerate}
    \item Join EIA generation data with EPA emissions data to calculate emissions rates in lbs/MWh.
    \item Calculate emissions from existing plants by taking model-outputted generation and multiplying that by the plant's share of cluster capacity to get a predicted MWh.  Then multiply that by the plant's emissions rates for all five pollutants.
    \item New fossil generation outputted by these models is only natural gas.  For new plants, I start by allocating capacity to both retired and existing sites, with a preference for siting at retired sites.
\end{enumerate}

https://ehp.niehs.nih.gov/doi/10.1289/EHP9001

What are emissions from hydrogen and natgas ccs?

\begin{singlespace}
\newpage
\bibliographystyle{jpe}
%\bibliographystyle{econometrica}
\bibliography{references.bib}
\end{singlespace}

\end{document}